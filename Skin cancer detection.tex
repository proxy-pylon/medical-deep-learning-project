\documentclass[11pt]{article}
\usepackage[a4paper,margin=1in]{geometry}
\usepackage{hyperref}

\title{Deep Learning for Automated Skin Cancer Detection}
\author{
	Adilet Akimshe,\\
	Rakhim Baimurzin,\\ 
	Lazzat Zhengisova,\\
	Alisher-Polat Kurmanayev,\\ 
	Yelzhas Omarov,\\
	Aidana Orazbay,\\
	Arailym Mukhametkali}
\date{}

\begin{document}
	\maketitle
	
	\section*{Problem \& Impact}
	We aim to build a system that maps dermoscopic and clinical skin images (plus optional metadata) to a diagnosis of malignant vs.\ benign lesion. Hypothesis: a convolutional neural network (CNN) trained on diverse, well-curated datasets can achieve dermatologist-level accuracy. This matters because early melanoma detection saves lives, but access to dermatologists is limited worldwide.
	
	\section*{Related Work}
	TODO
	
	\section*{Data}
	Primary source: ISIC Archive/HAM10000 (around 10k labeled dermoscopic images, CC-BY-NC). Optional additional clinical photos from other open repositories. Labels: benign/malignant and subtype. Planned split: 70/15/15 by patient ID. Preprocessing: resizing, color normalization; augmentation: flips, rotations, lighting variation.
	
	\section*{Method}
	Baseline: pretrained ResNet-50 fine-tuned on ISIC. Loss: weighted cross-entropy (to counter class imbalance). Plan: hyperparameter search (learning rate, weight decay, augmentation strength) via grid/random search. Explore multimodal fusion of metadata and image features.
	
	\section*{Evaluation}
	Metrics: ROC-AUC, PR-AUC, confusion matrix, and sensitivity at fixed specificity (90–95\%). Baselines: random, majority-class, pretrained ResNet. Target: ROC-AUC $>0.90$ on held-out test.
	
	\section*{Resources \& Reproducibility}
	Compute: student laptops or in the best case university GPU cluster (1–4 GPUs). Libraries: PyTorch, scikit-learn. Reproducibility: Git repo with fixed seeds, Docker/Conda env, dataset versioning on HuggingFace Datasets.
	
	\section*{Risks \& Ethics}
	Risks: (i) Class imbalance (mitigate with reweighting/augmentation); (ii) Patient-level leakage (mitigate by splitting by ID); (iii) Overfitting (early stopping, regularization). Ethics: avoid misuse as a diagnostic tool; dataset bias (under-representation of darker skin)
	
	\section*{Timeline \& Deliverables}
	Weeks 5–6: finalize data sources, preprocessing pipeline.  
	Weeks 7–8: implement baseline CNN, establish repo. interim demo (baseline results).  
	Weeks 9-10: advanced models + metadata fusion.  
	Week 11: evaluation and error analysis.  
	Week 12: final code, report, and poster submission.
	
	\bibliographystyle{plain}
	\begin{thebibliography}{9}
		https://github.com/proxy-pylon/medical-deep-learning-project
	\end{thebibliography}
	
\end{document}
